\documentclass[]{book}
\usepackage{amsmath,amsfonts,amssymb,amsthm}
\usepackage{hyperref}

\usepackage{color}
\newcommand{\taylor}[1]{{\color{blue} \sf $\spadesuit\spadesuit\spadesuit$ Taylor: [#1]}}
\newcommand{\anton}[1]{{\color{red} \sf $\spadesuit\spadesuit\spadesuit$ Anton: [#1]}}
\newcommand{\todo}[1]{{\color{purple} \sf $\spadesuit\spadesuit\spadesuit$ TODO: [#1]}}

\numberwithin{equation}{section}
\newtheorem{theorem}{Theorem}[subsection]
\newtheorem{lemma}[theorem]{Lemma}
\newtheorem{corollary}[theorem]{Corollary}
\newtheorem{proposition}[theorem]{Proposition}

\theoremstyle{definition}
\newtheorem{definition}[theorem]{Definition}
\newtheorem{question}[theorem]{Question}
\newtheorem{conjecture}[theorem]{Conjecture}
\newtheorem{example}[theorem]{Example}
\newtheorem{exercise}[theorem]{Exercise}

\theoremstyle{remark}
\newtheorem{remark}[theorem]{Remark}
\newtheorem{remarks}[theorem]{Remarks}
\newtheorem{warning}[theorem]{Warning}


\newcommand{\trdeg}{\operatorname{trdeg}}

\newcommand{\NN}{\mathbb{N}}
\newcommand{\ZZ}{\mathbb{Z}}
\newcommand{\QQ}{\mathbb{Q}}
\newcommand{\RR}{\mathbb{R}}
\newcommand{\CC}{\mathbb{C}}

\renewcommand{\AA}{\mathbb{A}}
\newcommand{\GG}{\mathbb{G}}
\newcommand{\PP}{\mathbb{P}}

\newcommand{\Spec}{\operatorname{Spec}}
\newcommand{\Ocal}{\mathcal{O}}

%\newcommand{\sec}{\operatorname{sec}}

\newcommand{\Aut}{\operatorname{Aut}}
\newcommand{\Hom}{\operatorname{Hom}}

\newcommand{\LM}{\operatorname{LM}}
\newcommand{\LT}{\operatorname{LT}}
\newcommand{\LC}{\operatorname{LC}}
\newcommand{\Low}{\operatorname{Low}}

\newcommand{\wt}{\operatorname{wt}}


%opening
\title{The Algebraic Theory of Differential Equations}
\author{Taylor Dupuy }


\begin{document}

\maketitle

\frontmatter

\tableofcontents

\chapter{FrontMatter}

\section*{Why do these note exist?}
These notes are from a course taught in Fall 2022 at UVM entitled \emph{The Algebraic Theory of Differential Equations}. 
I decided to write these notes because there are a lot different sources for Differential Algebra and I don't like any of them as an introduction to the subject for graduate students. 
They are either too old, or too specific, or too analytic, or too algebraic or model theoretic or non-rigorous.

For me the starting place is Ritt's book. 
I love Ritt's books \cite{Ritt1932} and \cite{Ritt1950} but they are old and hard to read. 
They are also missing a lot of the classical theory from the 1800s which motivated the subject. 
The successor to Ritt's books are Kolchin's books which, while mathematically very useful, invoke notation that gives me nightmares and has an obsession with removing intuitive analytic methods. 
Also, the algebraic geometry there largely ignores the development of scheme theory between from 1950 to 1970 by the French school. 
An alternative to these two is Kaplansky's book which I love but is perhaps too brief. 
The the classical theory of the hypergeometric functions and the Painlev\'{e} equations there is the classic book by Ince and 

\begin{itemize}
	\item There are also my advisor's books, which are probably the most influential on my perspective. 
	These are about differential fields and moduli problems, differential algebraic groups, and applications of differential algebra to Diophantine Geometry
	\item There are the seminal texts of Picard-Vessiot Theory (Galois theory for linear differential equations). 
	\item There is the more algebraic perpective by Deligne and Dwork which have a view towards applications to Arithmetic and Algebraic Geometry.
	\item There is the work on the Japanese school on both the theory of differential fields, so-called spaces of initial conditions for the Painlev\'{e} equations, and general differential Galois Theory.
	\item There is the work by the ``integrable systems'' community giving algebraic interpretations of famous solitonic equations (which also is closely tied to the work of the Japanese school and differential algebraic geometry). 
	\item There is work the Russian school which is largely focused on physical problems but often . 
	I personally had the pleasure as a student to take courses by  Zarkharov and Lushnikov.
	\item There is the work by Model Theorists which resolved what differentially closed fields should be?
\end{itemize}
There are a myriad of questions that can be address. 
My collaborator Jim Freitag's favorite: given a differential equation, what are the algebraic solutions one can possibly have?




 

Understandably, I can't cover this all. 
I'm not even going to pretend to try. 
My goal is to survey material.
Because of this, I'm going to need to assume some mathematics at times --- there already exists excellent references for much of the material we need to source.  
This will at times include basic Differential \cite{Ince1944} and Partial Differential Equations \cite{Evans2010}, Commutative Algebra \cite{Atiyah2016}, Galois Theory \cite{Cox2012}, Complex Analysis \cite{Ullrich2008}, Algebraic Topology \cite{Hatcher2002}, Manifolds \cite{Lee2013}, and Algebraic Geometry \cite{Vakil2017}. 
At the same time, I'm not crazy. 
I don't want to be writing to nobody. 
Things that I feel are part of a good introduction for well-prepared graduate students I will review. 

In addition to helping graduate students, I want to help  myself.
I have a number of things I would like to understand better. What is a $\tau$-function? What is a space of initial conditions? 
What is a Jacobian flow? What proofs work for differential equations but not for difference equations? 
What do we \emph{really} mean when we say $X$ equation is a limiting case of $Y$ equation?
What are the most fundamental examples to keep in mind and teach students when talking about this material?

%At the end of all of this there is going to be many course that could be taught using this book.



\section{Where can I get a digital copy of these notes?}
A link to the .tex can be found here:
\begin{center}
	\url{https://github.com/tdupu/diff-alg-public}.
\end{center}
I will be posting a Dropbox link on my webpage (at the time of writing this it is at \url{http://uvm.edu/~tdupuy}) but these always break. 
If you are taking the course and the link breaks let me know.

\mainmatter

\chapter{Hypergeometric Differential Equations}





\backmatter

\bibliographystyle{amsalpha}
\bibliography{diff-alg.bib}

\end{document}
